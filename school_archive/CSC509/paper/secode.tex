\chapter{Software Engineering Code of Ethics and Professional
Practice}\label{A:SECode}
\begin{center}
ACM/IEEE-CS Joint Task Force on Software Engineering Ethics and Professional
Practices
\end{center}

\section{PREAMBLE}
Computers have a central and growing role in commerce, industry, government,
medicine, education, entertainment and society at large. Software engineers are
those who contribute by direct participation or by teaching, to the analysis,
specification, design, development, certification, maintenance and testing of
software systems. Because of their roles in developing software systems,
software engineers have significant opportunities to do good or cause harm, to
enable others to do good or cause harm, or to influence others to do good or
cause harm. To ensure, as much as possible, that their efforts will be used for
good, software engineers must commit themselves to making software engineering
a beneficial and respected profession. In accordance with that commitment,
software engineers shall adhere to the following Code of Ethics and
Professional Practice.

The Code contains eight Principles related to the behavior of and decisions
made by professional software engineers, including practitioners, educators,
managers, supervisors and policy makers, as well as trainees and students of
the profession. The Principles identify the ethically responsible relationships
in which individuals, groups, and organizations participate and the primary
obligations within these relationships. The Clauses of each Principle are
illustrations of some of the obligations included in these relationships. These
obligations are founded in the software engineer�s humanity, in special care
owed to people affected by the work of software engineers, and the unique
elements of the practice of software engineering. The Code prescribes these as
obligations of anyone claiming to be or aspiring to be a software engineer.    

It is not intended that the individual parts of the Code be used in isolation
to justify errors of omission or commission. The list of Principles and Clauses
is not exhaustive. The Clauses should not be read as separating the acceptable
from the unacceptable in professional conduct in all practical situations. The
Code is not a simple ethical algorithm that generates ethical decisions. In
some situations standards may be in tension with each other or with standards
from other sources. These situations require the software engineer to use
ethical judgment to act in a manner which is most consistent with the spirit of
the Code of Ethics and Professional Practice, given the circumstances.

Ethical tensions can best be addressed by thoughtful consideration of
fundamental principles, rather than blind reliance on detailed regulations.
These Principles should influence software engineers to consider broadly who is
affected by their work; to examine if they and their colleagues are treating
other human beings with due respect; to consider how the public, if reasonably
well informed, would view their decisions; to analyze how the least empowered
will be affected by their decisions; and to consider whether their acts would
be judged worthy of the ideal professional working as a software engineer. In
all these judgments concern for the health, safety and welfare of the public is
primary; that is, the "Public Interest" is central to this Code.

The dynamic and demanding context of software engineering requires a code that
is adaptable and relevant to new situations as they occur. However, even in
this generality, the Code provides support for software engineers and managers
of software engineers who need to take positive action in a specific case by
documenting the ethical stance of the profession. The Code provides an ethical
foundation to which individuals within teams and the team as a whole can
appeal. The Code helps to define those actions that are ethically improper to
request of a software engineer or teams of software engineers.

The Code is not simply for adjudicating the nature of questionable acts; it
also has an important educational function. As this Code expresses the
consensus of the profession on ethical issues, it is a means to educate both
the public and aspiring professionals about the ethical obligations of all
software engineers.

\section{PRINCIPLES}

\subsection*{Principle 1: PUBLIC}
Software engineers shall act consistently with the public interest. In
particular, software engineers shall, as appropriate:

\begin{description}
\item[1.01.] Accept full responsibility for their own work.
\item[1.02.] Moderate the interests of the software engineer, the employer, the
client and the users with the public good.
\item[1.03.] Approve software only if they have a well-founded belief that it is
safe, meets specifications, passes appropriate tests, and does not diminish
quality of life, diminish privacy or harm the environment. The ultimate effect
of the work should be to the public good.
\item[1.04.] Disclose to appropriate persons or authorities any actual or
potential danger to the user, the public, or the environment, that they
reasonably believe to be associated with software or related documents.
\item[1.05.] Cooperate in efforts to address matters of grave public concern
caused by software, its installation, maintenance, support or documentation.
\item[1.06.] Be fair and avoid deception in all statements, particularly public
ones, concerning software or related documents, methods and tools.
\item[1.07.] Consider issues of physical disabilities, allocation of resources,
economic disadvantage and other factors that can diminish access to the
benefits of software.
\item[1.08.] Be encouraged to volunteer professional skills to good causes and
contribute to public education concerning the discipline.
\end{description}

\subsection*{Principle 2: CLIENT AND EMPLOYER}
Software engineers shall act in a manner that is in the best interests of their
client and employer consistent with the public interest.

\subsection*{Principle 3: PRODUCT}
Software engineers shall ensure that their products and related modifications
meet the highest professional standards possible.

\subsection*{Principle 4: JUDGEMENT}
Software engineers shall maintain integrity and independence in their
professional judgment.

\subsection*{Principle 5: MANAGEMENT}
Software engineering managers and leaders shall subscribe to and promote an
ethical approach to the management of software development and maintenance.

\subsection*{Principle 6: PROFESSION}
Software engineers shall advance the integrity and reputation of the profession
consistent with the public interest.

\subsection*{Principle 7: COLLEAGUES}
Software engineers shall be fair to and supportive of their colleagues.

\subsection*{Principle 8: SELF}
Software engineers shall participate in lifelong learning regarding the
practice of their profession and shall promote an ethical approach to the
practice of the profession.
