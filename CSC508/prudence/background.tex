\section{Background}
This research is concerned with the use of software in safety-critical systems.
Because of the explosive use of software in organizations today, safety risks
are at issue. A system is considered \textbf{safety-critical} if its ``failure
may cause injury ord eath to human beings'' \cite{FOLDOC}.

\subsection{Negligence Law}
Software developers build products just as any other engineer does. In the
safety-critical realm, software developers are subject to the same negligence
laws that apply in all of engineering. When, then, is negligence law invoked?

\begin{quote}
``\textbf{Negligence liability} attaches when injury or loss is caused by a 
failure to satisfy a duty that was imposed by law, as a matter of public
policy.''
\cite{Kaner_neg_1995}
\end{quote}

In general terms, an organization in the safety-critical realm has a duty to
take reasonable measures to ensure that its products are safe. The ideas of
\textit{duty} and \textit{reasonable} are ambiguous, but can be explicated. In
\cite{Kaner_neg_1995}, Cem Kaner adopts the forumula\footnote{this cost-benefit
analysis was originally expressed by Judge Learned Hand in \textit{United States
v. Carroll Towing Co.}} shown in Figure \ref{negligence} to qualitatively
evaluate how reasonable a company's actions were involving a safety-critical
product.
fqgwegads 
\begin{figure}
\begin{tabular}{|l|}
\hline
	Let \textbf{B} be the burden (expense) of preventing a potential accident.\\
	Let \textbf{L} be the severity of the loss if the accident occurs.\\
	Let \textbf{P} bet he probability of the accident.\\[6pt]
	Then \textit{failure to attempt to prevent a potential accident is 
	unreasonable if}\\[8pt]

      \makebox[\textwidth][c]{\(B < P \times L\)}
\\[3pt]
\hline
\end{tabular}
 
\label{negligence}
\caption{Negligence cost-benefit analysis}
\end{figure}

\newpage
An organization that develops safety-critical software, then, has a duty to
spend the amount of time and resources equivalent to the product of the severity
of harm and the likelihood that it will happen.

\subsection{Industry Standards}
\subsubsection{CMM}
The Cabability Maturity Model, or CMM, is concerned with improving processes.
The techniques laid out by CMM will help undisciplined organizations manage
software projects. By using the infrastructure in the CMM, organizations can
avoid software projects that finish late or overbudget.

\begin{quote}
``A \textit{software process} can be defined as a set of activities, methods,
practicies, and transformations that people use to develop and maintain software
and the associated products (e.g., project plans, design documents, code, test
cases, and user manuals).''
\end{quote}

The CMM characterizes five levels of maturity as follows.
\begin{enumerate}
  \item Level 1 -- The Initial Level
  \item Level 2 -- The Repeatable Level
  \item Level 3 -- The Defined Level
  \item Level 4 -- The Managed Level
  \item Level 5 -- The Optimizing Level
\end{enumerate}

For more information about CMM, please refer to \cite{CMM11}.
\subsubsection{ISO 9000-3}

