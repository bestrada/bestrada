\section{Background}
The emergence of computer software in society offers a new range of disputes and
misconduct. In recent times, software innovations have been the originating
cause for disputes in Internet identity theft \cite{Valetk2004}, digital
copyright infringement \cite{Manesh2006}, and death from malfunctioning
safety-critical software \cite{Leveson1993}. This research is concerned with the
use of software in safety-critical systems. Because of the explosive use of
software in organizations today, safety risks are at issue. A system is 
considered \textbf{safety-critical} if its ``\textit{failure may cause injury or
death to human beings}'' \cite{FOLDOC}.

\subsection{Tort Liability}
The term \textbf{tort} encompasses many disputes that involve wrongdoings, but 
can be expressed in general as ``\textit{an unwanted intrusion on a protected
personal right that causes physical, economic, or psychological injury}'' 
\cite{Burgunder2004}. Tort liability surfaces itself in the forms of strict
products liability and negligence law.

\subsubsection{Strict Products Liability}
Strict products liability focuses primarily on the condition of actual products
that are released to consumers, not the process used to develop them. Lee
Burgunder lays out that a merchant

\begin{quote}
``\ldots will be liable for damages resulting from an unreasonably dangerous
product, whether the seller was negligent or not.'' \cite{Burgunder2004}
\end{quote}

Personal injury is neccessary for the strict products liability standard to be
applied. When applied, defective products in question are subject to either
\textbf{manufacturing defect} consideration or \textbf{design defect}
consideration. The former applies when a product strays from the intended
design, while the latter pertains when the design of a product is not adequately
safe, even if the particular instance of product complies with the
manufacturer's intention \cite{Turner1999}. The application of strict products
liability against software is debatable, but is not of concern in this paper.

\subsubsection{Negligence Law}
A software developer builds products just as any other engineer does. In the
safety-critical realm, software developers are subject to the same negligence
laws that apply in all of engineering. Negligence law is concerned with behavior
that is socially unreasonable. When, then, is negligence law invoked?

\begin{quote}
``\textbf{Negligence liability} attaches when injury or loss is caused by a 
failure to satisfy a duty that was imposed by law, as a matter of public
policy.'' \cite{Kaner_neg_1995}
\end{quote}

Because duty is behavioral, negligence is only concerned with the
process\footnote{\textbf{software processes} are discussed further in
\ref{process} and \ref{sdp}} involved in creating the product, not the product
itself. In general terms, an organization in the safety-critical realm has a
duty to take reasonable measures to ensure that its products are safe. Software
is especially concerned with negligence law because it can be argued that
software itself is not really a product. Software can be described as
``\textit{a static description of a dynamic process}\footnote{quoted from Clark
S. Turner during a lecture of CSC508: Software Engineering}'' where a programmer
is providing the service of making a computer function as desired by a client.
This service is not sufficiently a tangible product\footnote{fundamental
properties of software are discussed further in \ref{software_props}}, so
software engineers fall under the jurisdiction of reasonable duties of care.

The ideas of \textit{duty} and \textit{reasonable} are ambiguous, but can be
explicated. Cem Kaner adopts the forumula\footnote{this cost-benefit analysis
was originally expressed by Judge Learned Hand inthe case of  \textit{United
States v. Carroll Towing Co.}} shown in Figure \ref{fig:negligence} to
qualitatively evaluate how reasonable a company's actions are.

\begin{figure}
\begin{tabular}{|l|}
\hline
	Let \textbf{B} be the burden (expense) of preventing a potential accident.\\
	Let \textbf{L} be the severity of the loss if the accident occurs.\\
	Let \textbf{P} be the probability of the accident.\\[6pt]
	Then \textit{failure to attempt to prevent a potential accident is 
	unreasonable if}\\[8pt]

      \centerline{\(B < P \times L\)}
\\[3pt]
\hline
\end{tabular}
\caption{Negligence cost-benefit analysis \cite{Kaner_neg_1995}}
\label{fig:negligence}
\end{figure}

An organization that develops safety-critical software, then, has a duty to
spend the amount of time and resources equivalent to the product of the severity
of harm and the likelihood that it will happen.

\subsection{Industry Standards}
There are several relevant standards that apply to the process model proposed in
this research. While the Software Engineering Institute, the International
Standards Organization, and NASA put forth standards to facilitate the entire
process of software engineering, we are only concerned with the 
\textit{documentation} efforts recommended by these guidelines.

\subsubsection{CMM}
The Cabability Maturity Model, or CMM, is concerned with improving processes.
According to the Software Engineering Institute, the process followed by an
organization exhibits that organization's maturity. The CMM describes:

\begin{quote}\label{process}
``A \textbf{software process} can be defined as a set of activities, methods,
practicies, and transformations that people use to develop and maintain software
and the associated products (e.g., project plans, design documents, code, test
cases, and user manuals).'' \cite{CMM11}
\end{quote}

The techniques laid out by CMM will help undisciplined organizations manage
software projects. By using the infrastructure in the CMM, organizations can 
avoid software projects that finish late or overbudget.

For more information about CMM as it applies to software, please refer to
\cite{CMM11}.

\subsubsection{ISO 9000-3}
The International Standards Organization put forth a quality management standard
known as ISO 9001 to describe a set of guidelines that will help organizations
achieve standards of quality that are recognized and respected throughout the
world. The ISO 9000-3 is the application of ISO 9001 to the field of software
engineering. They describe \textbf{software engineering} as: 

\begin{quote}
``\ldots a defined, step-by-step process that facilitates the specification,
design, implementation, and testing of a software solution for a set of stated
requirements in the most expeditious and cost-effective manner possible.''
\cite{Kehoe1996}
\end{quote}

During this process of software engineering, the ISO 9000-3 standard commands
that software development and maintenance efforts be based upon a defined
process that (among other things): 

\begin{quote}
``\ldots creates \textit{succinct and usable} level of documentation that will
support and facilitate continued enhancements to and maintenance of the product
as well as provide the basis for estimating future efforts.'' \cite{Kehoe1996}
\end{quote}

\subsubsection{NASA-STD-8719.13A}
The National Aeronautics and Space Administration publishes a set of guidelines
and standards that provide a methodology for software safety in their programs.
In \cite{NASA1997} specifically, their organization purposes describe the
activities necessary to ensure that safety is designed into and maintained
throughout the safety critical software lifecycle. NASA describes \textbf{safety
critical software} as software that\ldots

\begin{quote} 
``\ldots exercises direct command and control over the condition or state of
hardware components; and, if not performed, performed out-of-sequence, or
performed incorrectly could result in improper control functions (or lack of
control functions required for proper system operation), which could cause a
hazard or allow a hazardous condition to exist.'' \cite{NASA1997}
\end{quote}

Among other requirements, NASA demands of its software implementations that
safety-critical code ``shall be commented in such a way that future changes can
be made with a reduced likelihood of invoking a hazardous state.''
