\section{Previous Work}

\subsection{Legal Research}
The legal problems begot by software in safety-critical systems have already
been researched \cite{Turner1996, Turner2000} and some suggestions have been
made to help alleviate the cost of defending the risk associated with
safety-critical systems\cite{Turner2001}. Additionally, it has been identified
that software defects cannot easily be designated as manufacture-defective or
design-defective\cite{Turner2000}, and therefore cannot be classified as
strict-product liable or negligence liable in legal disputes. It is clear that
both aspects of liability be addressed. This research focuses on defects in the
process model vulnerable to negligence allegations.

There has already been work that tries to address this overall problem. Clark
Turner has researched the implications of safety-critical systems in
\cite{Turner1996, Turner2000, Turner2001} and as deduced that a popular
approach to this problem is the retroactive investigation of attorneys when a
legal dispute ensues. But this is both inefficient and oftentimes ineffective
because important historical information may be missing and unrecoverable.

\subsection{Documentation Solutions}

In \cite{Parnas1986}, David Parnas encourages documentation to compensate for
rational design. Since it is impossible to build a bug-free system, inadvertent
mistakes are forgivable with requirements and documentation. He lays out
acceptance criteria for ideal requirements documentation that can be applied to
safety-critical software environments. Common approaches include formal
requirement specifications, code comments, or more browseable documents like
Javadoc\cite{Javadoc}.

\subsubsection*{Code Comments}
This is inefficient because\ldots

\subsubsection*{Javadoc}
This is inefficient because\ldots

\subsection{Documentation Traceability}
Antoniol, et. al. researched techniques to recover traceability links between
code and free text documentation \cite{Antoniol2000}.
