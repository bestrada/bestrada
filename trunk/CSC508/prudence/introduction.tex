\section{Introduction}
Organizations today are in the business of producing better products faster,
cheaper, and more efficiently. As a result of society�s demands, these
organizations find themselves utilizing software for the benefits of rapid
deployment and efficiency over traditionally engineered physical components.
Safety critical systems are already beginning to see the use of software in
their implementations\cite{Leveson1993}, and disputes have occurred regarding
their adequacy. The advent of such systems juxtaposed with the complexities of
software breeds a new set of concerns that do not easily map to traditional
engineering standards.

Because of its unique nature, defects in software are inevitable and typically
more difficult to locate and handle than physical flaws in mechanical
components\cite{Brooks1987}. Defects in the safety-critical realm can be
especially dangerous. The use of software in safety-critical systems arouses
concern of product liability and negligence law.

\subsection{General Problem Statement}
What are the process requirements that an organization must undertake when
developing safety-critical software products?
