\begin{abstract}
Software in safety critical systems inevitably brings about the question of
negligence liability from defects that cause injury or harm. This paper examines
the fundamental inadequacies of software processes during development that
trigger defects and hinder maintainability. We specify an improvement to this
process model that overcomes these inadequacies and can be introduced into the
workflow of software development to help identify ``due process'' in a
safety-critical system and reduce the cost of defending such a system in the
case of a negligence dispute.\\

\noindent{\bf Index Terms}: documentation, comments, process constraints,
negligence, software safety, liability
\end{abstract}

\section{Introduction}

Organizations today are in the business of producing better products faster,
cheaper, and more efficiently. As a result of society�s demands, these
organizations find themselves utilizing software for the benefits of rapid
deployment and efficiency over traditionally engineered physical components.
Over the years, computers have progressed to the point where this is not only
possible, but more economically feasible than using solely hardware components. 
Sara Baase tracks technical development of computers and  shows the technical
growth that computers have seen since the early computing  days of the ENIAC in
Table \ref{tab:eniac_v_today} adopted from \cite{Baase1997}. Software has become
so available and accessible that previously impossible feats are now incredibly
easy to achieve.

Although software does  offer the benefits of added flexibility, increased
functionality, and reduced costs, it provides unprecedented possibilities for
errors. Safety critical systems are already beginning to see the use of software
in their implementations\cite{Leveson1993}, and disputes have occurred regarding
their adequacy. The advent of such systems juxtaposed with the complexities of 
software breeds a new set of concerns that do not easily map to traditional 
engineering standards.

\begin{table}
%extend the margins by 1.5 inches on both sides so this can be centered
\begin{narrow}{-1.5in}{-1.5in}
\begin{center}
\begin{tabular}{l|p{2in}|p{3in}}
$ $        & \centerline{$ENIAC$} & \centerline{$Today$} 
\\\hline
Speed      & 5,000 additions/sec   & More than 100,000,000 instructions/sec \\
Size       & 80 feet long, 30 tons & Notebook size to refrigerator size \\
Cost       & \$5--10 million (current dollars) & \$1,000 for a PC \\
Components & Vacuum tubes, resistors, switches & Integrated circuits, chips \\
Input Media & Punch cards & Keyboard, voice, scanners, handwriting, mouse, touch
screens \\
Output Media & Punch cards & On-screen text, graphics, and video; sound; laser
printers \\
Communications & None & Modems, fax, broadband; access to e-mail and the World
Wide Web \\
Software & Forget it. & You name it. \\
\end{tabular}
\end{center}
\end{narrow}
\caption{50 years -- Comparing the ENIAC with modern computers.}
\label{tab:eniac_v_today}
\end{table}

Because of its unique nature, defects in software are inevitable and typically
more difficult to locate and handle than physical flaws in mechanical
components\cite{Brooks1987}. Defects in the safety-critical realm can be
especially dangerous. The use of software in safety-critical systems arouses
concern of liability and negligence. In general, this research seeks to define a
prudent process strategy that an organization can undertake when developing
safety-critical software products.

Our approach proposes to use an enhanced commenting system that will allow
software developers to write free text comments at implementation time into a
centralized, traceable database. This will help legal analysts and other
developers link implementation details with actual developer intent at the time
of the program's writing. Doing so may help legal teams defend against
negligence liabilities.

Our paper is organized as follows. Section \ref{legal} describes the legal 
motivations of such a system as it relates to tort liability and negligence law.
Section \ref{previous} surveys various attempts to solve similar problems
related to the work that we do including industry process standards and
code-to-documentation linking. Section \ref{sdp} outlines the software
development process and where our solution would fit in the model of the
software development lifecycle. Section \ref{solution} describes our solution in
detail, including design goals and implementation considerations. In Section
\ref{results}, we explicate the potential of our solution, layout drawbacks, and
make suggestions for future work.

\subsection{General Problem Statement}
What measures can a safety-critical software organization undertake to help
reduce the cost of defending negligence liability in disputes and identify
social responsibility in its processes?
