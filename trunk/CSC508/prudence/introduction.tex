\begin{abstract}
Software in safety critical systems inevitably brings about the question of
negligence liability from product defects that cause injury or harm. This
paper examines the fundamental process inadequacies during the software
development process lifecycle that trigger defects in software. We specifies an
improvement to this process model that overcomes these inadequacies and can
be introduced into the workflow of sofware development to help reduce the losses
due to negligence liability.\\

\noindent{\bf Index Terms}: documentation, comments, safety-critical
\end{abstract}

\section{Introduction}

Organizations today are in the business of producing better products faster,
cheaper, and more efficiently. As a result of society�s demands, these
organizations find themselves utilizing software for the benefits of rapid
deployment and efficiency over traditionally engineered physical components. 
Sara Baase tracks technical development of computers and  shows the technical
growth that computers have seen since the early computing  days of the ENIAC in
Table \ref{tab:eniac_v_today}.

Although software does  offer the benefits of added flexibility, increased
functionality, and reduced costs, it provides unprecedented possibilities for
errors. Safety critical systems are already beginning to see the use of software
in their implementations\cite{Leveson1993}, and disputes have occurred regarding
their adequacy. The advent of such systems juxtaposed with the complexities of 
software breeds a new set of concerns that do not easily map to traditional 
engineering standards.

Because of its unique nature, defects in software are inevitable and typically
more difficult to locate and handle than physical flaws in mechanical
components\cite{Brooks1987}. Defects in the safety-critical realm can be
especially dangerous. The use of software in safety-critical systems arouses
concern of product liability and negligence law.

\begin{table}
%extend the margins by 1.5 inches on both sides so this can be centered
\begin{narrow}{-1.5in}{-1.5in}
\begin{center}
\begin{tabular}{l|p{2in}|p{3in}}
$ $        & \centerline{$ENIAC$} & \centerline{$Today$} 
\\\hline
Speed      & 5,000 additions/sec   & More than 100,000,000 instructions/sec \\
Size       & 80 feet long, 30 tons & Notebook size to refrigerator size \\
Cost       & \$5--10 million (current dollars) & \$1,000 for a PC \\
Components & Vacuum tubes, resistors, switches & Integrated circuits, chips \\
Input Media & Punch cards & Keyboard, voice, scanners, ahndwriting, mouse, touch
screens \\
Output Media & Punch cards & On-screen text, graphics, and vieo; sound; laser
printers \\
Communications & None & Modems, fax, broadband; access to e-mail and the World
Wide Web \\
Software & Forget it. & You name it. \\
\end{tabular}
\end{center}
\end{narrow}
\caption{50 years -- Comparing the ENIAC with modern computers. \cite{Baase1997}}
\label{tab:eniac_v_today}
\end{table}

\subsection{General Problem Statement}
What are the process requirements that an organization must undertake when
developing safety-critical software products?
