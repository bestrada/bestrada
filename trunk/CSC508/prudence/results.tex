\section{Results}
We have examined the legal implications of developing software for
safety-critical systems and deduced constraints on software processes in
accordance to these implications. In order for an organization to develop 
software prudently for use in safety-critical environments, they must engage in
certain activities. We describe one such activity: the commenting practices of
software developers.

In addition, we have designed the requirements for a system that will help
organizations fulfill some of the legal constraints for software organizations
in the safety-critical realm. If a system that implements our requirements were
in place, an organization will be able to provide adequate documentation of
design decisions during the software development process. Our system allows for
software developers to easily write versioned, multiple-associated comments in
real-time without severely disrupting the  development process or impeding the
readability of the code base.

\subsection{Drawbacks}

\subsubsection*{Documentation Attitude}
The success of this system requires a forward-thinking attitude of commitment to
documentation. Most domains of software engineering spend minimal time with
formal documentation methods in an effort to achieve faster turnaround and
short-term gain through quick (and often rushed) implementation cycles. This
disregard to the care of formal documentation may yield better immediate gains,
but ultimately degrades the quality of the implementation, reusability of the
software,  and traceability of design decisions.

\subsection{Future Work}

\subsection{Conclusions}