\chapter{Motivation}

Although software does offer the benefits of added flexibility, increased
functionality, and reduced costs, it provides unprecedented possibilities for
errors. Safety-critical systems have been on the bandwagon of using software in
their implementations for some time \cite{Graupe78,Hurtig94}, and disputes have
since then occurred regarding their stability \cite{Therac25,Maisel05}. The 
advent of such systems juxtaposed with the complexities of software breeds a new
set of concerns that do not easily map to traditional engineering standards.

Because of its unique nature, defects in software are inevitable and typically
more difficult to locate and handle than the physical flaws found in mechanical
components \cite{Parnas90}. Defects in software used in safety-critical
situations can be especially dangerous. The increasing use of software in
machines and the demand for more product functions adds complexity and more room
for error. Models to test, detect, and correct these errors exist and are
continually improving \cite{Parnas90}.

Currently, the software process does not explicitly take into account the legal
requirements that society demands of software engineers. Professor Clark S. 
Turner suggests that requirements due to products liability in tort can be
viewed as the requirements of ``remote customers'' of safety-critical consumer
software, which, as suggested in future work of \cite{Turner99}, 
``\textit{allows a direct and natural integration of these important 
considerations in the software process}''.

\section{Research Question}

Students of software engineering need to be aware of the societal effects that
the software they will write in industry is vulnerable to. How does our legal
system interact with the evolution and technical progress of software? What are
the integration points between the socially imposed legal requirements and our
software processes?

\section{Case Study: An Example}

In 2007, a software-defect related incident provoked a dispute between Upchurch
Plumbing, Inc. and Greenwood Utilities Commission\footnote{\textit{See: Upchurch
Plumbing, Inc. and Triconex Systems, Inc. v. Greenwood Utilities
Commission}, Westlaw Citation -- 2007 WL 1150903 (Miss.). The actual case is
listed in Appendix \ref{A:case}.}. The Municipal Energy
Agency of Mississippi (MEAM) contracted General Electric Company (GE) to upgrade
three generating units. With the consent of GE, MEAM contracted Greenwood
Utilities Comission (Greenwood) to one portion of the contract who, in turn,
contracted with Upchurch Plumbing, Inc. (Upchurch) to work on the control system
for its General Electric Frame V combustion turbine. Upchurch subcontracted with
Triconex Systems, Inc. (Triconex) to install a digital control system for the
turbine. This subcontract is referred to as Contract 103.

The dispute was over the combustion turbine. Engineers attempted to start the
turbine at the rated speed of 4,860 RPM but the mechanical overspeed bolt
tripped, shutting down the machine at a reported speed of 4,000 RPM -- according
to the Triconex digital controls. The engineers adjusted the mechanical  
overspeed trip bolt to allow faster speeds. It was later discovered, though,
that the machine was actually operating at a speed of 6,560 RPM instead of the
reported 4,860 RPM shown on the Triconex digital control system. The problem was
a programming defect in the software that utilized a 1:1.35 ratio which
corresponded to the auxiliary shaft instead of the appropriate 1:1 ratio used in
the turbine shaft. The Triconex software, therefore, reported speeds slower than
the actual operating speed of the turbine. When the engineers adjusted the
mechanical overspeed bolt, they thought the reported speeds were accurate. This
allowed the turbine to operate at dangerous speed of at least 6,932 RPM, causing
irreparable damage to the turbine.

The Leflore County Circuit Court in Mississippi ruled in favor of Greenwood and
awarding them over \$2.5 million in damages against Upchurch and Triconex.
Upchurch and Triconex appealed, but the Mississippi Supreme Court found no error
in judgement.

Though the case does not set a legal precedent for software engineers, it does
provide an example of a software-related incedent that costed on the scale of
millions of dollars. The court was unable to make a decision regarding products
liability because the holding of the trial court was based upon breach of
contract. Triconex and Upchurch were liable because they agreed to deliver a
working product to Greenwood in a written contract. They failed to do this and 
for this reason, the appellate court only addressed the breach of contract claim
and was unable to address negligence.

For most safety-critical products--those available to consumers and which are of
primary focus in this research--no actual contract exist. Negligence is based on
an unwritten, implied contract that exists because there is a duty of care that
is owed, in the case of safety-critical software, to the users of the software
products.
