\begin{center}
\section*{Silicon Valley Programmer Indicted for Manslaughter}
\subsection*{Program Error Caused Death by Robot}
by Mabel Muckraker\\
Special to the \textit{Silicon Valley Sentinel-Observer}\\
Silicon Valley, USA
\end{center}

Jane McMurdock, prosecuting attorney for the city of Silicon Valley, announced
today the indictment of Randy Samuels on charges of manslaughter. Samuels was
formerly employed as a programmer at Silicon Techtronics Inc., one of Silicon
Valley's newest entries into the high-tech arena. The charge involves the death
of Bart Matthews, who was killed last May by an assembly-line robot.

Matthews worked as a robot operator at Cybernetics Inc., in Silicon Heights. He
was crushed to death when the robot he was operating malfunctioned and started
to wave its ``arm'' violently. The robot arm struck Matthews, throwing him
against a wall and crushing his skull. Matthews died almost instantly. The case
has shocked and angered many in Silicon Valley. According to the indictment,
Samuels wrote the particular piece of computer program responsible for the
robot malfunction.

``There's a smoking gun!'' McMurdock announced triumphantly at a press conference
held in the Hall of Justice. ``We have the handwritten formula, provided by the
project physicist, which Samuels was supposed to program. But, he negligently
misinterpreted the formula, leading to this gruesome death. Society must
protect itself against programmers who make careless mistakes or else no one
will be safe, least of all our families and our children.''

The Sentinel-Observer has obtained a copy of the handwritten formula in
question. There are actually three similar formulas, scrawled on a piece of
yellow legal pad paper. Each formula describes the motion of the robot arm in
one direction: east-west, north-south and up-down. The Sentinel-Observer showed
the formulas to Bill Park, a professor of physics at Silicon Valley University.
He confirmed that these equations could be used to describe the motion of a
robot arm. The Sentinel-Observer then showed Park the program code, written by
the accused in the C programming language. We asked Park, who is fluent in C
and several other languages, whether the program code was correct for the given
robot-arm formulas.

Park's response was immediate. He exclaimed, ``By Jove! It looks like he
misinterpreted the y-dots in the formulas as y-bars, and he made the same
mistake for the x's and the z's. He was supposed to use the derivatives, but he
took the averages instead. He's guilty as hell, if you ask me.''

The Sentinel-Observer was unable to contact Samuels for comment. ``He is deeply
depressed about all this,'' his live-in girlfriend told us over the phone, ``but
Randy believes he will be acquitted when he gets a chance to tell his side of
the story.''
