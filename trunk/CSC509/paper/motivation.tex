\chapter{Motivation}\label{C:Motivation}

Although software does offer the benefits of added flexibility, increased
functionality, and reduced costs, it provides unprecedented possibilities for
errors. Safety-critical systems have been on the bandwagon of using software in
their implementations for some time \cite{Graupe78,Hurtig94}, and disputes have
since then occurred regarding their stability \cite{Leveson93,Maisel05}. The 
advent of such systems juxtaposed with the complexities of software breeds a new
set of concerns that do not easily map to traditional engineering standards.

Because of its unique nature, defects in software are inevitable and typically
more difficult to locate and handle than the physical flaws found in mechanical
components. Defects in software used in safety-critical situations can be
especially dangerous. The increasing use of software in machines and the demand
for more product functions adds complexity and more room for error. Models to
test, detect, and correct these errors exist and are continually improving.

Since testing is a key phase for software quality assurance, it is a clear
target for scrutiny under a legal dispute.

\section{General Problem Statement}
How does one qualitatively evaluate the burden or expense of preventing
potential hazardous conditions in safety-critical software systems?

\section{The Case of the Killer Robot}
``\textit{The Case of the Killer Robot}'' \cite{Epstein96} is a fictional  
scenario detailing software engineering with computer eithics that will be the
focus of observation in this paper. The hypothetical scenario consists of seven
newspaper articles, one journal article, and one magazine interview; each of
which are listed fully in Appendix \ref{A:Killer}.

\subsection{Facts}

\subsection{Issues}
