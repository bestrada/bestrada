\section{Products Liability}\label{S:Liability}
The term \textit{products liability} broadly applies to the liability of a
manufacturer or seller for injury to a buyer caused by a product that has been
sold \cite{Testing2005}. A \textit{product} usually refers to physical 
merchandise that can be purchased\footnote{\textit{Winter v. G.P. Putnam's
Sons}, 938 F.2d 1033.\\In this case, Wilhelm Winter became critically ill
from eating mushrooms that he picked relying on the information in a book
published by Putnam. The judge favored in the side of Putnam, claiming that the
contents of \textit{The Encyclopedia of Mushrooms} is not a product that can be
liable because the law does not take into consideration the unique
characteristics of ideas and expression. However, the plaintiff's argument was
strong when the book was analogized to aeronautical charts - graphical
depictions of technical and mechanical data. They are intended to be used while
engaging in hazardous activity. The discussion also mentions that
\textit{software} may be considered a product for this same reason. Software
that ``\textit{fails to yield the result for which it was designed}'' may be
considered under products liability}. Officially, a product is defined as 
\begin{quote}
``\textit{\ldots tangible personal property distributed commercially for use or
consumption and other items, such as real property and electricity, when the
context of their distribution and use is sufficiently analogous to the
distribution and use of tangible personal property\ldots}\footnote{Restatement
Third, Torts: Products Liability \S 19(a).}''
\end{quote}
In the case of safety-critical systems, software is usually embedded in
some machine or hardware device \cite{Leveson95} that is sold as a product
\footnote{59 A.L.R.5th 461.}. When viewed from this standpoint, software is less
analogous to pure thoughts and expressions and may be considered a product for
products liability cases. Many forms of products liability exist, including
\textit{Contract}, \textit{Strict}, and \textit{Negligence}.

\textit{Contracts}, as in software warranties or End User License Agreements
(EULAs), are issued to assure customers that the products purchased will perform
as stated \cite{Armour93}. Contract law can be dismissed because, as described
in the next section, negligence liability applies regardless of what terms are
steted in any contract. Even if a contract disclaims risk, manufacturers are
still held accountable and cannot absolve themselves from liability from
defects \cite{Ryan03}.

\textit{Strict Liability} applies to any product that is defective, regardless
of the amount of care used in the process to manufacture it\footnote{63 Am. Jur.
2d Products Liability \S 90.}. While strict liability is interesting
\cite{Turner00}, this research is concerned with the \textit{processes} and
\textit{tradeoff} analysis involved with developing and testing software. These
aspects of software engineering fall under the jurisdiction of negligence law.

\subsection{Negligence}\label{SS:Negligence}
We apply negligence law to our research\footnote{Negligence is 
concerned with process, not with product. The question is not whether software 
development can be applied to negligence law, but if negligence law can apply to
software development.}.

The legal term negligence refers to, in general, careless conduct. Scholars
describe negligence more specifically as:

\singlespace
\begin{itemize}
 \item the existence and violation of a legal duty to use care, proximately 
 causing injury to another.
 \item the failure to exercise the degree of care demanded by the circumstances.
 \item the breach of a duty to another to protect him or her from the particular
 harm that ensued.
 \item the want of that care the law prescribes under the particular
 circumstances existing at the time of the act or omission which is involved.
\end{itemize}
\doublespace

Negligence can be applied to many different scenarios beyond products
liability. An intoxicated driver who disobeys traffic laws is negligent towards
other citizens of the road. A teacher who fails to demonstrate safety
precautions to his students in wood shop may be liable for negligence. An
engineer who does not adequately inspect his high-integrity product can be
negligent to his clients. This research focuses on the negligence constrains as
they apply to products liability.

According to the Learned Hand test\footnote{\textit{United States v. Carroll
Towing Co.}, 159 F.2d 169.\\ Judge Learned Hand created this guideline to
determine the amount of duty owed in a negligence dispute. In the case, the
United States sought compensation for flour that was lost when a barge carrying
the cargo sunk. The barge company was partly responsible because no workers were
present on the barge when it sank, which may have prevented the barge from
sinking. Qualitatively, the amount it would have cost to keep a worker on the
barge would have been less than the product of the probability that the barge
sank and the amount of damages incurred from it sinking.}, shown in Figure
\ref{fig:handtest}, an organization that develops safety-critical software has a
duty to spend at least the amount of time and resources equivalent to the
product of the severity of harm and the likelihood that it will happen. If they
do not, then their actions are negligent.

\begin{figure}
\begin{narrow}{-1.5in}{-1.5in}\begin{center}
\begin{tabular}{|l|}
\hline
	Let \textbf{B} be the burden (expense) of preventing a potential accident.\\
	Let \textbf{L} be the severity of the loss if the accident occurs.\\
	Let \textbf{P} be the probability of the accident.\\[6pt]
	Then \textit{failure to attempt to prevent a potential accident is 
	unreasonable if}\\[8pt]

      \centerline{\(B < P \times L\)}
\\[3pt]
\hline
\end{tabular}
\end{center}\end{narrow}
\caption{The Learned Hand Test for Negligence}
\label{fig:handtest}
\end{figure}

\subsubsection{Elements of Negligence}\label{SS:Elements}
The applicability of negligence may be ambiguous and the laws of negligence can
only be invoked in certain situations. The prerequisites, or prima facie
elements, of a negligence case are \cite{Dobbs01}:

\singlespace
\begin{enumerate}
 \item there exists a duty of care (or duty to protect)
 \item the defendant breaches this duty with unreasonably risky conduct
 \item the defendant's conduct resulted in harm to the plaintiff
 \item the negligent conduct was a proximate cause of harm
 \item legally recognized damages or injury exist
\end{enumerate}
\doublespace

First, a negligence case calls for an actual duty of care owed to a plaintiff.
There may be a question about how much care is owed in a given situation, but
there are circumstances in which there is no duty owed that bears on the harm a
plaintiff suffers. Judges decide whether or not this duty exists \cite{Dobbs01}.

Also, there must be a breach of this duty of care owed. A defendant who behaves
reasonably and exercises the necessary care required by law will not be
negligent even if the plaintiff is harmed \cite{Dobbs01}.

The defendant's negligence must be the cause of the harm suffered by the
plaintiff. An careless engineer who does not test his product is not negligent
to the user who is injured by tripping over the machine. In addition, the cause
must not only be cause in fact, but a proximate, or primary, cause of the harm
suffered \cite{Dobbs01}.

Finally, actual damages or injuries must be suffered for a negligence case to
follow suit. This can include personal injury or damages to property
\cite{Dobbs01}.

\subsubsection{Software Fulfills the Prima Facie}

It is not a stretch to conclude that defective software in a safety-critical
situation will be subject to negligence allegations. The developing
organization clearly owes a duty of care to its customers. Since the software
will be used to perform tasks that can potentially cause harm, its users expect
a reasonably prudent amount of care from its developers.

The remaining elements are assumptions that this research seeks to evaluate.
Performing tests is a large part of quality assurance for software and doing it
correctly can mitigate the risk of unreasonably breaching a duty of care.
