\chapter{Background}
The analysis of the legality of testing processes for software demands a
discussion of the constraints laid out by the law. Topics comprised in products
liability and negligence applicably lays out the obligations of software
engineers in the safety-critical realm.

The term \textit{products liability} broadly applies to the liability of a
manufacturer or seller (defendant) for injury to a buyer (plaintiff) caused by
a product that has been sold. While many forms of products liability exist, we
apply negligence law to our research.

\section{Negligence}
The legal term negligence refers to, in general, careless conduct. Scholars
describe negligence more specifically as:

\singlespace
\begin{itemize}
 \item the existence and violation of a legal duty to use care, proximately 
 causing injury to another.
 \item the failure to exercise the degree of care demanded by the circumstances.
 \item the breach of a duty to another to protect him or her from the particular
 harm that ensued.
 \item the want of that care the law prescribes under the particular
 circumstances existing at the time of the act or omission which is involved.
\end{itemize}
\doublespace

Negligence can be applied to many different scenarios beyond products
liability. An intoxicated driver who disobeys traffic laws is negligent towards
other citizens of the road. A teacher who fails to demonstrate safety
precautions to his students in wood shop may be liable for negligence. An
engineer who does not adequately inspect his high-integrity product can be
negligent to his clients. This research focuses on the negligence constrains as
they apply to products liability.

\subsection{Elements of Negligence}
The applicability of negligence may be ambiguous and the laws of negligence can
only be invoked in certain situations. The prerequisites, or prima facie
elements, of a negligence case are:

\singlespace
\begin{enumerate}
 \item there exists a duty of care (or duty to protect)
 \item the defendant breaches this duty with unreasonably risky conduct
 \item the defendant�s conduct resulted in harm to the plaintiff
 \item the negligent conduct was a proximate cause of harm
 \item legally recognized damages or injury exist
\end{enumerate}
\doublespace

First, a negligence case calls for an actual duty of care owed to a plaintiff.
There may be a question about how much care is owed in a given situation, but
there are circumstances in which there is no duty owed that bears on the harm a
plaintiff suffers.

Also, there must be a breach of this duty of care owed. A defendant who behaves
reasonably and exercises the necessary care required by law will not be
negligent even if the plaintiff is harmed.

The defendant's negligence must be the cause of the harm suffered by the
plaintiff. An careless engineer who does not test his product is not negligent
to the user who is injured by tripping over the machine. In addition, the cause
must not only be cause in fact, but a proximate, or primary, cause of the harm
suffered.

Finally, actual damages or injuries must be suffered for a negligence case to
follow suit. This can include personal injury or damages to property.

\subsection{Software Fulfills the Prima Facie}

It is not a stretch to conclude that defective software in a safety-critical
situation will be subject to negligence allegations. The developing
organization clearly owes a duty of care to its customers. Since the software
will be used to perform tasks that can potentially cause harm, its users expect
a reasonably prudent amount of care from its developers.

The remaining elements are assumptions that this research seeks to evaluate.
Performing tests is a large part of quality assurance for software and doing it
correctly can mitigate the risk of unreasonably breaching a duty of care.

\section{Legal Requirements}

It has been established that software developers in the safety-critical realm
owe a duty of care to the intended users of their software. Part of this duty
of care is the manufacturer's duty to test its products. Applicable law states
that:

\begin{quote}
a manufacturer has a duty to test and inspect its products, at least where the
nature of the product is such that an injury from its use is foreseeable if the
product is defective, and where tests or inspections are practicable and would
be effective.''
\end{quote}

Since safety-critical software can cause injury, it is subject to this
stipulation. But as previously mentioned, the unique characteristics of
software offer new problems in testing that make it difficult to create a
mapping with traditional methods of engineering and the legal requirements for
testing. These legal requirements include ensuring the product is operable for
the purpose intended, securing the production of a finished and safe article,
using the most effective methods known and available, and testing and
inspecting during and after the course of manufacture.

The safety of testing processes are legally scrutinized over the following four
non-functional metrics: 

\begin{description}
\item[Verifiability] the software should be verified against its own
requirements. Is the product reasonably fit for its intended purposes? Did the
tests match the severity of the product�s expected use? Verifiability measures
how well the end software article complies with the requirements set up before
development.
\item[Completeness] the tests and inspections of the software should exhibit
complete coverage to the extent that is reasonably feasible. Were the methods
used conclusive and effective for demonstrating the products safety?
Completeness measures how thorough the tests were done for exhibiting product
safety.
\item[Credibility] the testing process used should be at least on par with what
is considered the state-of-the-art by other reasonably prudent software
engineers. Did the manufacturer exercise the most effective known or best
available methods in their test strategy? Credibility refers to the conduct of
the testing manner customarily followed by others in the industry.
\item[Maintainability] software tests should not be a single stage of
development, but a pervasive process exercised throughout the product�s
lifecycle. Did the manufacturer perform tests even after the product was
considered �complete�? If defects were found after the fact, did they release
fixes? Maintainability measures how well the manufacturer preserves the quality
of their product throughout its life.
\end{description}
