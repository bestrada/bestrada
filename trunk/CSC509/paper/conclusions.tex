\chapter{Conclusions}

The work produced in this research is by no means sufficient for prudent testing
processes. Meeting sufficient conditions would guarantee a necessary condition.
Enumerating all of the measures to take to guarantee a software engineer's duty
of care is fulfilled would be an impossible task. This research establishes a
minimum criteria for software testing in relation to the social constraints
imposed by the law. Compliance with these measures will not absolve any party
from their legal obligations.

We argue, however, that in safety critical software systems the measures laid
out in our research are absolutely necessary, though not sufficient, for
software testing. Logical equivalence designates that not taking these
measures implies negligence and prudence implies taking these measures.

\section{Weaknesses and Future Work}
Due to limitations in time, resources, and background knowledge, the legal
portions of this research may not be of the highest quality. This paper is a
work-in-progress, and the authors maintain to improve the quality of legal
research and writing.

Time and access constraints also limited the standards available to apply to
social constraints. Future work will include references to the latest ISO 9000-3
standards, but for the time being, Table \ref{thetable} refers exclusively to
IEEE Std 1012-2004.

Unfortunately, there is no strong and quantitative metric to validate the
success of this research. Instead, we satisfice \cite{Simon96} with a solution
that is helpful enough for audiences to learn about social constraints and
discover possible approaches to satisfying them. Future work will be better
validated through with industry acceptance surveys through qualitative
interviews and site visits to real safety-critical software organizations.
