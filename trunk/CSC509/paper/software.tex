\chapter{Software and the Law}

\section{Legal Requirements}

It has been established that software developers in the safety-critical realm
owe a duty of care to the intended users of their software. Part of this duty
of care is the manufacturer's duty to test its products. Applicable law states
that:

\begin{quote}
a manufacturer has a duty to test and inspect its products, at least where the
nature of the product is such that an injury from its use is foreseeable if the
product is defective, and where tests or inspections are practicable and would
be effective.''
\end{quote}

Since safety-critical software can cause injury, it is subject to this
stipulation. But as previously mentioned, the unique characteristics of
software offer new problems in testing that make it difficult to create a
mapping with traditional methods of engineering and the legal requirements for
testing. These legal requirements include ensuring the product is operable for
the purpose intended, securing the production of a finished and safe article,
using the most effective methods known and available, and testing and
inspecting during and after the course of manufacture.

The safety of testing processes are legally scrutinized over the following four
non-functional metrics: 

\begin{description}
\item[Verifiability] the software should be verified against its own
requirements. Is the product reasonably fit for its intended purposes? Did the
tests match the severity of the product's expected use? Verifiability measures
how well the end software article complies with the requirements set up before
development.
\item[Completeness] the tests and inspections of the software should exhibit
complete coverage to the extent that is reasonably feasible. Were the methods
used conclusive and effective for demonstrating the products safety?
Completeness measures how thorough the tests were done for exhibiting product
safety.
\item[Credibility] the testing process used should be at least on par with what
is considered the state-of-the-art by other reasonably prudent software
engineers. Did the manufacturer exercise the most effective known or best
available methods in their test strategy? Credibility refers to the conduct of
the testing manner customarily followed by others in the industry.
\item[Maintainability] software tests should not be a single stage of
development, but a pervasive process exercised throughout the product's
lifecycle. Did the manufacturer perform tests even after the product was
considered ``complete''? If defects were found after the fact, did they release
fixes? Maintainability measures how well the manufacturer preserves the quality
of their product throughout its life.
\end{description}