\chapter{Software Testing and the Law}\label{C:Software}
Legal doctrine dictates the care that an organization tests and inspects the
products that it develops. This chapter will examine the legal requirements for
testing software, then apply activities and standards that can fulfill these
requirements.

\section{Legal Requirements for Testing Software}

It is arguable that software developers in the safety-critical realm owe a duty
of care to the intended users of their software. Ultimately, judges make the
decision as to whether or not a duty of care is owed in the first place
\cite{Dobbs01} and this research assumes that this criterion is met. That
stated, part of this duty of care is the manufacturer's duty to test its
products. Applicable law states that:

\begin{quote}
``\textit{a manufacturer has a duty to test and inspect its products, at least
where the nature of the product is such that an injury from its use is 
foreseeable if the product is defective, and where tests or inspections are 
practicable and would be effective.}'' \cite{Testing2005}
\end{quote}

Since safety-critical software can cause foreseeable injury, it is subject to
this duty to test. But as previously mentioned, the unique characteristics of
software offer new problems in testing that make it difficult to create a
mapping with traditional methods of engineering and the legal requirements for
testing. These problems will become apparent during the discussion of legal
requirements. The legal requirements for testing in general include insuring the
product is operable for the purpose intended, securing the production of a
finished and safe article, using the most effective methods known and available,
and testing and inspecting during and after the course of manufacture
\cite{Testing2005}.

First, the law demands that a manufacturer test its product to be reasonably fit
for its intended purpose\footnoteremember{Huff_v_White}{\textit{Huff v. White Motor Corp.},
565 F.2d 104.\\In this case, the court reversed its previous decision
regarding ``intended purposes''. In the original case, Jesse Huff died from
severe burns suffered from an accident that happened while driving a 
truck-tractor manufactured by White Motor Corporation. The fuel tank ruptured 
and engulfed the cab, occupied by Huff, in flames. The court's original decision
held that the plaintiff (Huff's wife), had no claim for relief because the
faulty design did not cause or contribute to injury (it enhanced the injury 
received) and being involved in a collision is not included in the automobile's
``intended purposes''. In the appellate court, this decision was reversed. The 
original construction of ``intended purposes'' was too narrow and unrealistic. 
Borrowing from a previous decision\footnotemark, the court determined that the injuries suffered
from accident are ``readily foreseeable as an incident to the normal and
expected use of an automobile''.}.
\footnotetext{\textit{Larsen v. General Motors Corp.}, 391 F.2d 495.} This is a
difficult judgement to make in software. How does one determine what is
``reasonably fit''? Who even determines the ``intended purposes'' of of a piece
of software?

The law also measures the tests performed by a software development team against
that of a ``reasonably prudent manufacturer under the same or similar 
circumstances\footnote{\textit{Schenck v. Pelkey}, 176 Conn. 245,
405 A.2d 665, 25 U.C.C Rep. Serv. 416, 6 A.L.R.4th 481.\\Russel Schenck charged
the Pelkey family (and later, the manufacturer Goshen) with negligence after
sustaining quadriplegic injuries from sliding head-first down a pool slide.
Since there were no standards for pool slides, the court correctly charged the
jury to decide if a ``reasonably-prudent corporation'' under the same 
circumstances would have conducted tests for such an injury.}'' This stipulation
is of particular importance for software engineers because, unlike other forms
of engineering, there are no standards for software testing processes to be
judged against\footnoteremember{sw-license}{An article in IEEE Software answers
common questions about standards for licensing software engineers \cite{Moore03}
and work is being done to make this happen \cite{Knight00}.}. How does one
determine what a prudent manufacturer would do? What is it that constitutes a
``prudent'' software developer?

Finally, the law demands that a manufacturer perform adequate testing both
during and after the process of manufacture\footnote{\textit{Nicklaus v. Hughes
Tool Company}, 417 F.2d 983.}. This is important because software is an
always-evolving artifact. Software is quick to change and easy to alter, so
adequate testing must be a pervasive effort.

\subsection{Itemization}
When evaluating the legality of potentially negligent manufacturing processes,
specifically relating to how the product was tested, the following questions
should be asked:

\singlespace
\begin{itemize}
  \item Is the product reasonably fit for its intended purposes?
  \item Were the methods used conclusive and effective for demonstrating the
  product's safety?
  \item Did the manufacturer exercise the most effective known or best available
  methods in their test strategy?
  \item Did the manufacturer adequately perform tests during the course of
  manufacture \textit{and} after the product was considered ``complete''?
\end{itemize}
\doublespace

This list is not intended to be exhaustive, but is an abridgement of focused
legal considerations that should be made when judging an organization's testing
processes. Our approach does not produce explicit answers to these questions,
but instead organizes these legal constraints into the life-cycle stages and
verification activities according to \cite{CSUR-testing}. We then provide
suggested standards of practice to follow that may fulfill these process
requirements.

\section{The Justice in Testing Software}
Much of software quality is achieved through testing. Though product testing is 
traditionally applied after manufacture, the abstraction of software allows (and
arguably necessitates) that testing occur on a more iterative and flexible
fashion \cite{Boehm86}. Though software has exponentially evolved and improved,
the methods of validation, verification, and testing have not changed much. The
survey \cite{CSUR-testing} of this area designates key life-cycle stages that we
use to designate legal requirements and standards of practice.

\subsection{Requirements}
Though it occurs in the very beginning, the requirements definition stage of
development also contains verification, validation, and testing activities that
are important to the process as a whole. During the requirements definition
stage, software engineers gather requirements and validate that they are 
actually the correct and appropriate requirements as stated (or implied) by the
intended user. In addition, the team generates test data to be used in future
stages.

In terms of social constraints, courts will ask several questions related to
the requirements stage. Are the requirements as laid out reasonably fit for the
intended purposes of the software? Did the organization prepare test data that
is commensurate with the severity of the expected use of the software? 
Requirements testing insures that software engineers took these factors into
account.


\subsection{Design}
The tests and inspections of the software should exhibit complete coverage to
the extent that is reasonably feasible. Completeness measures how thorough the
tests were done for exhibiting product safety. The law holds that such tests be
thorough and complete to the point of adequately demonstrating the product's
safety.

\subsection{Construction}
the testing process used should be at least on par with what is considered the
state-of-the-art by other reasonably prudent software engineers. Credibility
refers to the conduct of the testing manner customarily followed by others in
the industry. There are no real industry accepted standards for testing or
software processes in general\footnoterecall{sw-license}, so juries must decide
if the processes used for testing software are credible in comparison to prudent
software engineering.

\subsection{Operation and Maintenance}
software tests should not be a single stage of development, but a pervasive
process exercised throughout the product's lifecycle. If defects were found
after the fact, did they release fixes? Maintainability measures how well the 
manufacturer preserves the quality of their product throughout its life.

\section{Satisfying the \textit{B}}

Unfortunately, there is no silver bullet \cite{Brooks87} to building quality
software. There is no real way for us to provide sufficient conditions that will
guarantee prudent behavior or explicitly qualify the expense taken to prevent
harm. What we can provide is a hierarchy of organized social constraints that
may be useful when trying to ensure due care and list possible practices that
can fulfill these legal requirements.
\singlespace
\begin{longtable}{| p{2in} | p{2in} | p{2in} |}
\caption{Activities, Constraints, and Standards}\label{thetable}\\

\hline
\textbf{Verification Activity} & \textbf{Social Constraints} & \textbf{Standard or Practice}\\
\hline
\endhead

\hline \multicolumn{3}{r}{{Continued on next page}} \\ \hline
\endfoot

\hline \hline
\endlastfoot

\multicolumn{3}{|c|}{\textit{1. Requirements}}\\
\hline
a. Determine verification approach & \multirow{2}{2in}{Insure that the product is reasonably fit
for its intended purposes} & \multirow{2}{2in}{TODO1}\\
\cline{1-1}
b. Determine adequacy of requirements & & \\
\hline
c. Generate functional test data & Tests and inspections match the severity of
the product�s expected use & TODO2\\
\hline
\multicolumn{3}{|c|}{\textit{2. Design}}\\
\hline
a. Determine consistency of design with requirements & Insure that the product
is reasonably fit for its intended purposes & TODO3\\
\hline
b. Determine adequacy of design & Utilize most practical and technically
feasible alternative design & TODO4\\
\hline
c. Generate structural and functional test data & Tests and inspections match
the severity of the product�s expected use & TODO5\\
\hline
\multicolumn{3}{|c|}{\textit{3. Construction}}\\
\hline
a. Determine consistency with design & Insure that the product is reasonably
fit for its intended purposes & TODO6\\
\hline
b. Determine adequacy of implementation & Secure the production of a safe
product & TODO7\\
\hline
c. Generate structural and functional test data & Tests and inspections match
the severity of the product�s expected use & TODO8\\
\hline
d. Apply test data & Conclusive and effective to demonstrate the product�s
safety & TODO9\\
\hline
\multicolumn{3}{|c|}{\textit{4. Operation and Maintenance}}\\
\hline
\multirow{4}{2in}{a. Reverify, commensurate with the level of redevelopment} & Make such tests and inspections during and after the process of manufacture &
TODO10\\
\cline{2-3}
 & Test and inspect at the time(s) and place where the tests and inspections
will be effective & TODO11\\
\cline{2-3}
 & Discover latent hazards involved in the use of the product & TODO12\\
\cline{2-3}
 & Conduct tests and inspections following changes in the construction of the
product and/or field complaints of the defects & TODO13\\
\hline
\end{longtable}
\doublespace