\chapter{Software Testing and the Law}\label{C:Software}

\section{Legal Requirements for Testing Software}

It is arguable that software developers in the safety-critical realm owe a duty
of care to the intended users of their software. Ultimately, judges make the
decision as to whether or not a duty of care is owed in the first place
\cite{Dobbs01} and this research assumes that this criterion is met. That
stated, part of this duty of care is the manufacturer's duty to test its
products. Applicable law states that:

\begin{quote}
``\textit{a manufacturer has a duty to test and inspect its products, at least
where the nature of the product is such that an injury from its use is 
foreseeable if the product is defective, and where tests or inspections are 
practicable and would be effective.}'' \cite{Testing2005}
\end{quote}

Since safety-critical software can cause foreseeable injury, it is subject to
this duty to test. But as previously mentioned, the unique characteristics of
software offer new problems in testing that make it difficult to create a
mapping with traditional methods of engineering and the legal requirements for
testing. These problems will become apparent during the discussion of legal
requirements. The legal requirements for testing in general include ensuring the
product is operable for the purpose intended, securing the production of a
finished and safe article, using the most effective methods known and available,
and testing and inspecting during and after the course of manufacture
\cite{Testing2005}.

First, the law demands that a manufacturer test its product to be reasonably fit
for its intended purpose\footnoteremember{Huff_v_White}{\textit{Huff v. White Motor Corp.},
565 F.2d 104.\\In this case, the court reversed its previous decision
regarding ``intended purposes''. In the original case, Jesse Huff died from
severe burns suffered from an accident that happened while driving a 
truck-tractor manufactured by White Motor Corporation. The fuel tank ruptured 
and engulfed the cab, occupied by Huff, in flames. The court's original decision
held that the plaintiff (Huff's wife), had no claim for relief because the
faulty design did not cause or contribute to injury (it enhanced the injury 
received) and being involved in a collision is not included in the automobile's
``intended purposes''. In the appellate court, this decision was reversed. The 
original construction of ``intended purposes'' was too narrow and unrealistic. 
Borrowing from a previous decision\footnotemark, the court determined that the injuries suffered
from accident are ``readily foreseeable as an incident to the normal and
expected use of an automobile''.}.
\footnotetext{\textit{Larsen v. General Motors Corp.}, 391 F.2d 495.} This is a
difficult judgement to make in software. How does one determine what is
``reasonably fit''? Who even determines the ``intended purposes'' of of a piece
of software?

The law also measures the tests performed by a software development team against
that of a ``reasonably prudent manufacturer under the same or similar 
circumstances\footnote{\textit{Schenck v. Pelkey}, 176 Conn. 245,
405 A.2d 665, 25 U.C.C Rep. Serv. 416, 6 A.L.R.4th 481.\\Russel Schenck charged
the Pelkey family (and later, the manufacturer Goshen) with negligence after
sustaining quadriplegic injuries from sliding head-first down a pool slide.
Since there were no standards for pool slides, the court correctly charged the
jury to decide if a ``reasonably-prudent corporation'' under the same 
circumstances would have conducted tests for such an injury.}'' This stipulation
is of particular importance for software engineers because, unlike other forms
of engineering, there are no standards for software testing processes to be
judged against\footnoteremember{sw-license}{An article in IEEE Software answers
common questions about standards for licensing software engineers \cite{Moore03}
and work is being done to make this happen \cite{Knight00}.}. How does one
determine what a prudent manufacturer would do? What is it that constitutes a
``prudent'' software developer?

Finally, the law demands that a manufacturer perform adequate testing both
during and after the process of manufacture\footnote{\textit{Nicklaus v. Hughes
Tool Company}, 417 F.2d 983.}. This is important because software is an
always-evolving artifact. Software is quick to change and easy to alter, so
adequate testing must be a pervasive effort.

\subsection{Itemization}
When evaluating the legality of potentially negligent manufacturing processes,
specifically relating to how the product was tested, the following questions
should be asked:

\singlespace
\begin{itemize}
  \item Is the product reasonably fit for its intended purposes?
  \item Were the methods used conclusive and effective for demonstrating the
  product's safety?
  \item Did the manufacturer exercise the most effective known or best available
  methods in their test strategy?
  \item Did the manufacturer adequately perform tests during the course of
  manufacture \textit{and} after the product was considered ``complete''?
\end{itemize}
\doublespace

This list is not intended to be exhaustive, but is an abridgement of focused
legal considerations that should be made when judging an organization's testing
processes. Our approach does not produce explicit answers to these questions,
but instead provides a more detailed set of questions that can be used to
evaluate the actual answers to these questions.

\section{The Justice in Testing}

There are many aspects of testing software that satisfy these legal
requirements. When analyzing the legality of an organizations testing processes,
the following considerations should be made:

\begin{description}
\item[Verifiability] the software should be verified against its own 
requirements. The software's commodity considers if it is appropriate for it's
intended use. Did the tests match the severity of the product's expected
use? Verifiability measures how well the end software article complies with the
requirements set up before development.
\item[Completeness] the tests and inspections of the software should exhibit
complete coverage to the extent that is reasonably feasible. Completeness
measures how thorough the tests were done for exhibiting product safety.
\item[Credibility] the testing process used should be at least on par with what
is considered the state-of-the-art by other reasonably prudent software
engineers. Credibility refers to the conduct of
the testing manner customarily followed by others in the industry.
\item[Maintainability] software tests should not be a single stage of
development, but a pervasive process exercised throughout the product's
lifecycle. If defects were found after the fact, did they release
fixes? Maintainability measures how well the manufacturer preserves the quality
of their product throughout its life.
\end{description}

\subsection{Verifiability}
In software engineering, validation and verification test ensure that software
engineers take into account the appropriate intended purpose of the product and
that the implementation actually safely fulfills this purpose.

\subsection{Completeness}
The law also holds that such tests be thorough and complete to the point of
adequately demonstrating the product's safety. 

\subsection{Credibility}

Software suffers from this same problem. There are no real industry accepted
standards for testing or software processes in
general\footnoterecall{sw-license}, so juries must decide if the processes used
for testing software are credible in comparison to prudent software engineering. 

\subsection{Maintainability}

\section{Evaluating the \textit{B}}

Unfortunately, there is no silver bullet \cite{Brooks87} to building quality
softare. There is no real way for us to provide sufficient conditions that will
guarantee prudent behavior or explicitly qualify the expense taken to prevent
harm. What we can provide is a heirarchy of organized social constraints that
may be useful when trying to ensure due care and list possible practices that
can fulfill these legal requirements.

\singlespace
\begin{enumerate}
  \item Is the software reasonably fit for its intended purposes?
  \begin{enumerate}
    \item Did the organization create valid requirements?
    \item Did the implementation verify against the software's own requirements?
  \end{enumerate}
  \item Were the testing methods reasonable for determining safety?
    \begin{enumerate}
    \item Did the coverage of tests match the severity of use that the
    software is expected to encounter?
    \end{enumerate}
  \item Did the manufacturer use the best methods known or available?
    \begin{enumerate}
      \item What programming paradigms did the developers employ (e.g.
      Object-Oriented, Aspect-Oriented, Procedural) and were they appropriate
      for the application?
      \item What testing processes did the software company employ?
      \item Did the testing process employ current best practices like unit,
      integration, and system testing?
    \end{enumerate}
  \item Did the manufacturer perform tests during and after the course of
  manufacture? 
\end{enumerate}
\doublespace

\begin{center}
\begin{table}
\caption{Software Testing and the Law}
\begin{tabular}{|p{1in}|p{2.5in}|p{2.5in}|}
\hline
\textbf{Maxim} & \textbf{Law or Social Constraint} & \textbf{Standard or Practice} \\[3pt]
\hline
\multicolumn{3}{|l|}{\textit{Reasonably fit for intended purposes}} \\
\hline
valid requirements & dude & sweet \\
verified implementation & dude & sweet \\
\hline
\multicolumn{3}{|l|}{\textit{Resonable testing methods}} \\
\hline
matched severity & dude & sweet \\
\hline
\multicolumn{3}{|l|}{\textit{Best methods known or available}} \\
\hline
programming paradigm & dude & sweet \\
testing processes & dude & sweet \\
unit test & dude & sweet \\
integration test & dude & sweet \\
system test & dude & sweet \\
\hline
\end{tabular}
\label{table2}
\end{table}
\end{center}