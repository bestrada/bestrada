\chapter{Introduction}

Historically, the liability of software has not been an issue of
concern for three reasons. First, software was originally used only by experts
and corporations for limited and purposed uses beyond the reaches of end
users and the general public \cite{Leveson95}. Second, processor instruction 
sets were primitive and software programs were very simple \cite{Storey95}. 
Finally, the availability of attorneys with applicable expertise to handle 
software cases has been rare \cite{Armour93}. Today's society no longer observes
these limitations.

Organizations today are in the business of building better products faster,
cheaper, and more efficiently. As a result of society's demands, these
organizations find themselves increasingly relying on software for the benefits
of rapid deployment and efficiency over traditionally engineered physical
components. Over the years, software has almost become an ideal product.
Software can be created quickly and easily. There are little to no material
costs associated with producing software and there is no deterioration over
time. It can be replicated cheaply once written and, since the inception of the
Internet, it is more mutable (e.g. updateable) than ever. By now, computers
have progressed to the point where using software in solutions not only
possible, but more economically feasible than using solely hardware components 
\cite{Baase97}. Software is oftentimes the only feasible alternative for solving
certain problems.

Complexity of software programs has also increased. Significant work in
programming only began in the 1950s, and mainstream software use didn't begin
until recently. But over time, knowledge of software has grown exponentially
and with developments like higher-level programming languages, object-oriented
design, and open source libraries, the complexity of software has equally
increased.

Thirdly, the number of qualified attorneys is growing. The American Bar 
Association's computer law division has grown from 89 members in 1980 to over 
1100 in 1993 \cite{Armour93}. Representation of computer science and other
related undergraduate majors in law schools is also growing.

Software devices continue to be used in risky situations. Microprocessors and
software are being used to control prosthetic limbs \cite{Graupe78}, to help
restore sight sight to the blind \cite{Fox95}, to automate the braking systems
in cars \cite{Hurtig94}, and even to treat cancer patients with radiation
\cite{Leveson93}. Because of the rising use of software in industry, its
increased complexity in implementation, and growth of qualified professionals,
the risk of liability will continue to appear. But society is still unprepared
to handle this risk and evaluate the methods behind software development.

We focus our research on software testing. Are the processes that we use to test
software legally sufficent in fulfilling our responsibilities as professionals?
How does one \textit{decide} whether or not they are? After laying a foundation
of knowledge in Chapter \ref{C:Background}, we can tightly define this 
``\textit{deciding}'' problem in Chapter \ref{C:Motivation} and propose a solution
in Chapter \ref{C:Software}.
