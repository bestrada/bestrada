\chapter{Background}\label{C:Background}
The analysis of the legality of testing processes for software demands a
discussion of constraints. The first part of this chapter examines the current
practices in software engineering testing processes We then look at the laws
that apply to these processes.

\section{Current Practices in Software Testing}
Despite the evidence of ineffective safety assurance \cite{Leveson93,Maisel05},
current standards in the software industry illustrate that efforts are being
made to ensure the safety of mission critical software applications.


\subsection{ISO 9000-3}
The International Standards Organization put forth a quality management standard
known as ISO 9001 to describe a set of guidelines that will help organizations
achieve standards of quality that are recognized and respected throughout the
world. The ISO 9001 guidelines are not specifically designed for safety-critical
software (or even software at all), but provide models for quality assurence
in the design, development, production, installation, and servicing of systems
in general \cite{Kehoe96}.

The ISO 9000-3 is the application of ISO 9001 the development, supply, and
maintenence of software. In terms of testing, the ISO 9000-3 standard adopts a
phase approach, similar to the waterfall model explained in \cite{Royce70}. The
six phases, at a high level, are:
\singlespace
\begin{enumerate}
  \item System Engineering/System Analysis
  \item Software Requirements Analysis
  \item Design
  \item Implementation
  \item Testing
  \item Maintenance
\end{enumerate}
\doublespace
Acording to ISO 9000-3, testing is a pervasive effort, taking place during the
entiere product-development cycle. For example: during the requirements phase, a
team must develop a preliminary test plan and identify test cases. During the
design phase, the team develops test cases and updates the test plan. During
implementation, system-level test cases are populated and unit-level test cases
are executed. And during the test phase, the tem conducts functional,
integration, and system tests and performs regression testing as needed. For
more information, refer to \cite{Kehoe96}.

\subsection{CMM}

\subsection{IEEE/ANSI Standards}

\section{Negligence}\label{S:Negligence}
The term \textit{products liability} broadly applies to the liability of a
manufacturer or seller for injury to a buyer caused by a product that has been
sold. While many forms of products liability exist\footnote{Other forms of
liability that safety-critical software developers risk are \textit{Strict
Liability} and \textit{Breach of Contract} liabilities. \textit{Strict Liability}
applies to any product that is defective, regardless of the amount of care used
in the process to manufacture it. \textit{Contract} law applies when a 
manufacturer's product fails to match the promises made in some written 
contract.}, we apply negligence law to our research\footnote{Negligence is 
concerned with process, not with product. The question is not whether software 
development can be applied to negligence law, but if negligence law can apply to
software development.}.

The legal term negligence refers to, in general, careless conduct. Scholars
describe negligence more specifically as:

\singlespace
\begin{itemize}
 \item the existence and violation of a legal duty to use care, proximately 
 causing injury to another.
 \item the failure to exercise the degree of care demanded by the circumstances.
 \item the breach of a duty to another to protect him or her from the particular
 harm that ensued.
 \item the want of that care the law prescribes under the particular
 circumstances existing at the time of the act or omission which is involved.
\end{itemize}
\doublespace

Negligence can be applied to many different scenarios beyond products
liability. An intoxicated driver who disobeys traffic laws is negligent towards
other citizens of the road. A teacher who fails to demonstrate safety
precautions to his students in wood shop may be liable for negligence. An
engineer who does not adequately inspect his high-integrity product can be
negligent to his clients. This research focuses on the negligence constrains as
they apply to products liability.

According to the Learned Hand test\footnote{This evaluation originated in the
case of \textit{United States v. Carroll Towing Co.} (159 F.2d 169) by Judge
Learned Hand who created this guideline to determine the amount of duty owed.
In the case, the United States sought compensation for flour that was lost when
a barge carrying the cargo sunk. The barge company was partly responsible
because no workers were present on the barge when it sank, which may have
prevented the barge from sinking. Qualitatively, the amount it would have costed
to keep a worker on the barge would have been less than the product of the
probability that the barge sank and the amount of damages incurred from it
sinking.}, shown in Figure \ref{fig:handtest}, an organization that develops 
safety-critical software has a duty to spend at least the amount of time and
resources equivalent to the product of the severity of harm and the likelihood 
that it will happen. If they do not, then their actions are negligent.

\begin{figure}
\begin{narrow}{-1.5in}{-1.5in}\begin{center}
\begin{tabular}{|l|}
\hline
	Let \textbf{B} be the burden (expense) of preventing a potential accident.\\
	Let \textbf{L} be the severity of the loss if the accident occurs.\\
	Let \textbf{P} be the probability of the accident.\\[6pt]
	Then \textit{failure to attempt to prevent a potential accident is 
	unreasonable if}\\[8pt]

      \centerline{\(B < P \times L\)}
\\[3pt]
\hline
\end{tabular}
\end{center}\end{narrow}
\caption{The Learned Hand Test for Negligence}
\label{fig:handtest}
\end{figure}

\subsection{Elements of Negligence}\label{SS:Elements}
The applicability of negligence may be ambiguous and the laws of negligence can
only be invoked in certain situations. The prerequisites, or prima facie
elements, of a negligence case are \cite{Dobbs01}:

\singlespace
\begin{enumerate}
 \item there exists a duty of care (or duty to protect)
 \item the defendant breaches this duty with unreasonably risky conduct
 \item the defendant's conduct resulted in harm to the plaintiff
 \item the negligent conduct was a proximate cause of harm
 \item legally recognized damages or injury exist
\end{enumerate}
\doublespace

First, a negligence case calls for an actual duty of care owed to a plaintiff.
There may be a question about how much care is owed in a given situation, but
there are circumstances in which there is no duty owed that bears on the harm a
plaintiff suffers. Judges decide whether or not this duty exists \cite{Dobbs01}.

Also, there must be a breach of this duty of care owed. A defendant who behaves
reasonably and exercises the necessary care required by law will not be
negligent even if the plaintiff is harmed \cite{Dobbs01}.

The defendant's negligence must be the cause of the harm suffered by the
plaintiff. An careless engineer who does not test his product is not negligent
to the user who is injured by tripping over the machine. In addition, the cause
must not only be cause in fact, but a proximate, or primary, cause of the harm
suffered \cite{Dobbs01}.

Finally, actual damages or injuries must be suffered for a negligence case to
follow suit. This can include personal injury or damages to property
\cite{Dobbs01}.

\subsection{Software Fulfills the Prima Facie}

It is not a stretch to conclude that defective software in a safety-critical
situation will be subject to negligence allegations. The developing
organization clearly owes a duty of care to its customers. Since the software
will be used to perform tasks that can potentially cause harm, its users expect
a reasonably prudent amount of care from its developers.

The remaining elements are assumptions that this research seeks to evaluate.
Performing tests is a large part of quality assurance for software and doing it
correctly can mitigate the risk of unreasonably breaching a duty of care.

