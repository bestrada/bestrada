\section{Intended Audience}

This paper is not written specifically for technical audiences. As will be
discussed in later chapters, the goal of this research is to assist those who
need to evaluate the amount of care that is put in to developing a software
product by the organization that developed it.

Judges, attorneys, and juries will find this work useful. Since software is a 
complex concept to grasp, this work will help non-experts understand the 
processes involved in creating and testing such an intangible artifact.

The primary audience, however, consists of software engineers, particularly
those involved in developing software products that may be harmful to their
users\footnote{Developers of medical implant software, aircraft and vehicle
software, or heavy equipment software are included in this group. Though it is
always good to prudently test your software applications, those who write code
for spreadsheet applications, cell-phones, and media players will not find this
work particularly useful.}. The first principle of the Software Engineering Code
of Ethics states that ``\textit{software engineers shall act consistently with
the public interest}'' \cite{SECODE}. Our goal is help improve the quality (and
thereby safety) of systems that use software in their implementations. Software
engineers who work in this realm may use this research to help them tighten
their testing processes and ensure that their quality assurance methods are, at
the very least, on par with the methods described in this paper.
