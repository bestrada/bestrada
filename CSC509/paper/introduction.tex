\chapter{Introduction}

Historically, the product liability of software has not been an issue of
concern for three reasons. First, software was originally used only by experts
and corporations for limited and purposed uses � beyond the reaches of end
users and the general public. Second, processor instruction sets were primitive
and software programs were very simple. Finally, the availability of attorneys
with applicable expertise to handle software cases has been rare. Today�s
society no longer observes these limitations.

Organizations today are in the business of building better products faster,
cheaper, and more efficiently. As a result of society�s demands, these
organizations find themselves increasingly relying on software for the benefits
of rapid deployment and efficiency over traditionally engineered physical
components. Over the years, software has almost become an ideal product.
Software can be created quickly and easily. There are little to no material
costs associated with producing software and there is no deterioration over
time. It can be replicated cheaply once written and, since the inception of the
Internet, it is more mutable (e.g. updateable) than ever. By now, computers
have progressed to the point where using software in solutions not only
possible, but more economically feasible than using solely hardware components.
Software is oftentimes the only feasible alternative for solving certain
problems.

Complexity of software programs have also increased. Significant work in
programming only began in the 1950s, and mainstream software use didn�t begin
until recently. But over time, knowledge of software has grown exponentially
and with developments like higher-level programming languages, object-oriented
design, and open source libraries, the complexity of software has equally
increased.

In addition, the number of qualified attorneys is growing. The American Bar
Association�s computer law division has grown from 89 members in 1980 to over
1100 in 1993. Representation of computer science and other related
undergraduate majors in law schools is also growing.

If the use of software continues to grow as it has been, then disputes over its
quality and the developers� liability will increasingly become apparent in the
legal world.

\section{Why This Problem is Important}
Although software does offer the benefits of added flexibility, increased
functionality, and reduced costs, it provides unprecedented possibilities for
errors. Safety-critical systems have been on the bandwagon of using software in
their implementations for some time, and disputes have since then occurred
regarding their stability. The advent of such systems juxtaposed with the
complexities of software breeds a new set of concerns that do not easily map to
traditional engineering standards.

Because of its unique nature, defects in software are inevitable and typically
more difficult to locate and handle than the physical flaws found in mechanical
components. Defects in software used in safety-critical situations can be
especially dangerous. The increasing use of software in machines and the demand
for more product functions adds complexity and more room for error. Models to
test, detect, and correct these errors exist and are continually improving.

Since testing is a key phase for software quality assurance, it is a clear
target for scrutiny under a legal dispute.

\section{General Problem Statement}
Are the current standards of procedure for software testing processes legally
sufficient �reasonable� quality control practices for detecting and correcting
harmful defects in safety-critical software and mitigating liability exposure?
