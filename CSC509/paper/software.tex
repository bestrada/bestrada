\chapter{Software Testing and the Law}\label{C:Software}
Legal doctrine dictates the care that an organization tests and inspects the
products that it develops. This chapter will examine the legal requirements for
testing software, then apply activities and standards that can fulfill these
requirements.

\section{Legal Requirements for Testing Software}

It is arguable that software developers in the safety-critical realm owe a duty
of care to the intended users of their software. Ultimately, judges make the
decision as to whether or not a duty of care is owed in the first place
\cite{Dobbs01} and this research assumes that this criterion is met. That
stated, part of this duty of care is the manufacturer's duty to test its
products. Applicable law states that:

\begin{quote}
``\textit{a manufacturer has a duty to test and inspect its products, at least
where the nature of the product is such that an injury from its use is 
foreseeable if the product is defective, and where tests or inspections are 
practicable and would be effective.}''\footnote{Am. Law. Prod. Liab. 3d, Chapter
11, \S 11:1}
\end{quote}

Since safety-critical software can cause foreseeable injury, it is subject to
this duty to test. But as previously mentioned, the unique characteristics of
software offer new problems in testing that make it difficult to create a
mapping with traditional methods of engineering and the legal requirements for
testing. These problems will become apparent during the discussion of legal
requirements. The legal requirements for testing in general include insuring the
product is operable for the purpose intended, securing the production of a
finished and safe article, using the most effective methods known and available,
and testing and inspecting during and after the course of 
manufacture\footnote{Am. Law. Prod. Liab. 3d, Chapter 11}.

First, the law demands that a manufacturer test its product to be reasonably fit
for its intended purpose\footnoteremember{Huff_v_White}{\textit{Huff v. White Motor Corp.},
565 F.2d 104.\\In this case, the court reversed its previous decision
regarding ``intended purposes''. In the original case, Jesse Huff died from
severe burns suffered from an accident that happened while driving a 
truck-tractor manufactured by White Motor Corporation. The fuel tank ruptured 
and engulfed the cab, occupied by Huff, in flames. The court's original decision
held that the plaintiff (Huff's wife), had no claim for relief because the
faulty design did not cause or contribute to injury (it enhanced the injury 
received) and being involved in a collision is not included in the automobile's
``intended purposes''. In the appellate court, this decision was reversed. The 
original construction of ``intended purposes'' was too narrow and unrealistic. 
Borrowing from a previous decision\footnotemark, the court determined that the injuries suffered
from accident are ``readily foreseeable as an incident to the normal and
expected use of an automobile''.}. This is a difficult judgement to make in
software. How does one determine what is ``reasonably fit''? Who even determines
the ``intended purposes'' of of a piece of software?
\footnotetext{\textit{Larsen v. General Motors Corp.}, 391 F.2d 495.} 

The law also measures the tests performed by a software development team against
that of a ``reasonably prudent manufacturer under the same or similar 
circumstances\footnote{\textit{Schenck v. Pelkey}, 176 Conn. 245,
405 A.2d 665, 25 U.C.C Rep. Serv. 416, 6 A.L.R.4th 481.\\Russel Schenck charged
the Pelkey family (and later, the manufacturer Goshen) with negligence after
sustaining quadriplegic injuries from sliding head-first down a pool slide.
Since there were no standards for pool slides, the court correctly charged the
jury to decide if a ``reasonably-prudent corporation'' under the same 
circumstances would have conducted tests for such an injury.}'' This stipulation
is of particular importance for software engineers because, unlike other forms
of engineering, there are no standards for software testing processes to be
judged against\footnoteremember{sw-license}{An article in IEEE Software answers
common questions about standards for licensing software engineers \cite{Moore03}
and research has been done to investigate this \cite{Knight00}.}. How does one
determine what a prudent manufacturer would do? What is it that constitutes a
``prudent'' software developer?

Finally, the law demands that a manufacturer perform adequate testing both
during and after the process of manufacture\footnote{\textit{Nicklaus v. Hughes
Tool Company}, 417 F.2d 983.}. This is important because software is an
always-evolving artifact. Software is quick to change and easy to alter, so
adequate testing must be a pervasive effort.

\subsection{Itemization}
When evaluating the legality of potentially negligent manufacturing processes,
specifically relating to how the product was tested, the following questions
should be asked:

\singlespace
\begin{itemize}
  \item Is the product reasonably fit for its intended purposes?
  \item Were the methods used conclusive and effective for demonstrating the
  product's safety?
  \item Did the manufacturer exercise the most effective known or best available
  methods in their test strategy?
  \item Did the manufacturer adequately perform tests during the course of
  manufacture \textit{and} after the product was considered ``complete''?
\end{itemize}
\doublespace

This list is not intended to be exhaustive, but is an abridgement of focused
legal considerations that should be made when judging an organization's testing
processes. Our approach does not produce explicit answers to these questions,
but instead organizes these legal constraints into the life-cycle stages and
verification activities according to \cite{CSUR-testing}. We then provide
suggested standards of practice to follow that may fulfill these process
requirements.

\section{The Justice in Testing Software}
Much of software quality is achieved through testing. Though product testing is 
traditionally applied after manufacture, the abstraction of software allows (and
arguably necessitates) that testing occur on a more iterative and flexible
fashion \cite{Boehm86}. Though software has exponentially evolved and improved,
the methods of validation, verification, and testing have not changed much. The
survey \cite{CSUR-testing} of this area designates key life-cycle stages
(Requirements, Design, Construction, and Operations and Maintenance) that we
use to correlate legal requirements\footnote{mostly adopted from Am. Law. Prod.
Liab. 3d and Am. Jur. 2d Products Liability} and standards of 
practice\footnote{from listings in \cite{CMM11,IEEE-std-verif}} to different
areas of verification, validation, and testing software. We also apply our
analysis to the hypothetical described in Section \ref{killer} and listed
fully in Appendix \ref{A:Killer}.

\subsection{Requirements}
Though it occurs in the very beginning, the requirements definition stage of
development also contains verification, validation, and testing activities that
are important to the process as a whole. During the requirements definition
stage, software engineers gather requirements and validate that they are 
actually the correct and appropriate requirements as stated (or implied) by the
intended user. In addition, the team generates test data to be used in future
stages.

\subsubsection*{Social Constraints}
In terms of social constraints, courts will ask several questions related to
the requirements stage. Are the requirements as laid out reasonably fit for the
intended purposes of the software?\footnote{Am. Law. Prod. Liab. 3d, Chapter 11,
\S 11:1} Did the organization prepare test data that
is commensurate with the severity of the expected use of the
software?\footnote{Am. Jur. 2d Products Liability \S 319} 
Requirements testing insures that software engineers took these factors into
account.

\subsubsection*{Standards or Practices}
Requirements activities standards span across multiple areas of activity.
Determining the verification approach is handled by \textit{test plan
generation} and determining the adequacy of requirements is handled by
\textit{software requirements evaluation}. To generate functional test data that
matches the severity of the software's expected use, a prudent software engineer
would use \textit{criticality analysis}. \textit{Criticality analysis} is the
process of determining the level of criticality that the piece software can be
classified as and then deciding the rigor and intensity to be applied to
verification and validation tasks\footnote{\cite{IEEE-std-verif} \S 4}. The IEEE
Std 1012-2004 suggests that software engineers perform this criticality analysis
throughout most of the activities involved in the development process.

\subsubsection*{The Case of the Killer Robot}
The Killer Robot case is arguably good in the requirements area. Silicon 
Techtronics created a seemingly prudent requiments document. As described in
article five, the requirements document states: ``\textit{The robot will be safe
to operate and even under exceptional conditions (see Section 5.2) the robot
will cause no bodily injury to the human operator.}''

However, Silicon Techtronics may not have been as successful in 
generating test data and scenarios that match the severity of use. The
requirements also state that ``\textit{\ldots the human operator will be able to
enter a sequence of command codes\ldots which will arest robot motion long
before bodily injury can actually occur.}'' This requirement, though well-meant,
may not have been a good way to implement the safety feature. According to the
article, Matthews was ``\textit{\ldots confused and could not enter the codes
successfully.}''

\subsection{Design}
During the design stage sofware architects actually frame the development work
that needs to be done on a project. To test these efforts, software organizations
should acquire tools for validation support and develop test procedures. The
software organization should also generate data for use in tests for later
activities. The design itself should also be analyzed for adequacy by using
walk-throughs or inspections. Requirements should be validated against user
needs.

\subsubsection*{Social Constraints}
\subsubsection*{Standards or Practices}
Section 5.4.3 of IEEE Std 1012-2004 commits software developers to address
software architectural design and continue test planning. The standard states
that Design V\&V endeavors to demonstrate that a design is correct, accurate,
and complete. Consistency with requirements is achieved through software design
evaluation and interface analysis and later in component test execution. Hazard,
security, and risk analysis are tasks that occur throughout every activity in
the development process to determine the adequacy of the software's design. And
test plan and test design generation are both tasks that are part of the
software's design activities.

\subsubsection*{The Case of the Killer Robot}
In terms of design, the User Interface (UI) of the Robbie CX30 was arguably
faulty. When held to a ``reasonable person standard'', Silicon Techtronics
designers did not behave in accordance to how a reasonably prudent manufacturer
would. UI expert Horace Gritty in article six, the UI, though perfectly implemented
according to the design, was poorly designed and violated Ben Shneiderman's
eight golden rules of user interface design, a well-read subject for computer
interface experts.

To list these design violations quickly, the CX30 UI: was inconsistent with the
presentation of error messages, did not allow shortcuts for frequent users, did
not provide feedback for the completion of certain dangerous actions, made error
recovery ``\textit{tedious, frustrating, and at times infuriating}'', did not
allow easy reversal of actions, presented a low feeling of user control over the
system, and contained large menus that relied too heavily on the user's memory.

\subsection{Construction}
Actual code execution with test data generated from the requirements and design
stages occurs during construction. Software engineers should verify the
implementation against the design and requirements, generate test data, and
actually apply the data to test cases. Specific techniques are as important as
choosing the correct technique according to what is appropriate for the software
system being tested.

\subsubsection*{Social Constraints}
\subsubsection*{Standards or Practices}
Most software construction verification and validation tasks will occur during
the implementation activity, according to IEEE Std 1012-2004. This stage of V\&V
sets out to verify that the transformation from design to implementation is
correct, accurate, and complete. A traceability analysis will determine
consistency with design. Test case and test procedure generation will set up an
organization for test activities later. Integration, system, and acceptence
test execution (from the plans made in this and previous stages) will determine
the adequacy of the software's implementation when applied to test data.

\subsubsection*{The Case of the Killer Robot}
The processes used to construct and execute realistic test data may have been
unethical in this scenario. The primary argument against good judgement in terms
of testing during the construction stage has to do with Silicon Techtronics'
tradeoff decisions. The team decided to release the product without it passing
all test cases in order to make more money from a sooner release date. Tester
Cindy Yardlay confirms this in her email to Robotics Division Chief Ray Johnson
when she writes:

\begin{quote}
``\textit{I have finished creating the software test results for that
troublesome robot software, as per your idea of using a simulation rather than
the actual software. Attached you will find the modified test document, showing
the successful simulation.}''
\end{quote}

The tradeoff was justified by the UI that would supposedly allow the operator to
halt the robot's motion. By not investing in the repairs that would secure the
passing of all tests, Silicon Techtronics may not have spent enough resource to
justify the risk of an accident happening.

\subsection{Operation and Maintenance}
Operation and maintenance stages occur after the software product is
``complete'' according to the development plan. But a software organization is
still responsible for installing, testing, and maintaining the software at the
client's site. Modifications and adjustments will, as is normally the case,
occur after installation and the developers are responsible for regression
testing commensurate with the changes made.

\subsubsection*{Social Constraints}

\subsubsection*{Standards or Practices}
Operations and maintenance usually occur at a customer site to test and inspect
the software product running in its target environment. In order to reverify the
software product commensurate with the level of changes made from the transition
to the production environment, the IEEE Std 1012-2004 instructs testers to
perform an installation configuration audit and installation checkout. Hazard
analysis should also occur throughout all activities in the development process.

Beyond the development process, there is also a short process on operation
within the IEEE Std 1012-2004. Tasks and activities in this process satisfy the
legal constraint to test following changes in the construction of the product
and field complaints of defects.

\subsubsection*{The Case of the Killer Robot}
The articles in ``\textit{The Case of the Killer Robot}'' reported no violations
in maintenence tasks other than than the company's failure to adequately train
end-users of their software. This activity belongs in the maintenance and
operations stage of development, and Silicon Techtronics ``\textit{only spent
one work day of approximately eight hours}'' to train operators while the
requirements state that that ``\textit{Only employees of the customer who have
passed [certifying tests] shall be allowed to operate the Robbie CX30 robot in
an industrual setting.}''

\section{Satisfying the \textit{B}}

Unfortunately, there is no silver bullet \cite{Brooks87} to building quality
software. The onus of providing a $B$ that is greater than or equal to $P \times
L$ is on the developer. But There is no real way for us to provide sufficient
conditions that will guarantee prudent behavior or explicitly qualify the
expense taken to prevent harm. What we can provide is a hierarchy of organized
social constraints that may be useful when trying to ensure that due care is met
and list possible practices that can fulfill these legal requirements.

Table \ref{thetable} classifies judicious behavior of software engineers
according to the activities presented in \cite{CSUR-testing}, relates them with
social constraints obtained from the law according to \cite{Prosser}, and
suggests tasks that may satisfice from \cite{CMM11} and \cite{IEEE-std-verif}.
\newpage
\singlespace
\begin{longtable}{| p{2in} | p{2in} | p{2in} |}
\caption{Activities, Constraints, and Standards}\label{thetable}\\

\hline
\textbf{Verification Activity} & \textbf{Social Constraints} & \textbf{Standard or Practice}\\
\hline
\endhead

\hline \multicolumn{3}{r}{{Continued on next page}} \\ \hline
\endfoot

\hline \hline
\endlastfoot

\multicolumn{3}{|c|}{\textit{\textbf{1. Requirements}}}\\
\hline
a. Determine verification approach & \multirow{2}{2in}{Insure that the product is reasonably fit
for its intended purposes} & \underline{IEEE Std 1012-2004}:
\begin{small}\begin{itemize}
\item System test plan generation - 5.4.2(5)
\item Acceptance test plan generation - 5.4.2(6)
\end{itemize}\end{small}\\
\cline{1-1} \cline{3-3}
b. Determine adequacy of requirements & & \underline{IEEE Std 1012-2004}:
\begin{small}\begin{itemize}
\item Software requirements evaluation - 5.4.2(2)
\end{itemize}\end{small}\\
\hline
c. Generate functional test data & Tests and inspections match the severity of
the product�s expected use & \underline{IEEE Std 1012-2004}:
\begin{small}\begin{itemize}
\item Criticality analysis - 5.4.3(4)
\item Component test plan generation - 5.4.3(5)
\item Integration test plan generation - 5.4.3(6)
\item Component test design generation - 5.4.3(7)
\item Integration test design generation - 5.4.3(8)
\item System test design generation - 5.4.3(9)
\item Acceptance test design generation - 5.4.3(10)
\end{itemize}\end{small}\\
\hline \newpage
\multicolumn{3}{|c|}{\textit{\textbf{2. Design}}}\\
\hline
a. Determine consistency of design with requirements & Insure that the product
is reasonably fit for its intended purposes & \underline{IEEE Std 1012-2004}:
\begin{small}\begin{itemize}
\item Traceability analysis - 5.4.3(1)
\item Software design evaluation - 5.4.3(2)
\item Interface analysis - 5.4.3(3)
\end{itemize}\end{small}\\
\hline
b. Determine adequacy of design & Utilize most practical and technically
feasible alternative design & \underline{IEEE Std 1012-2004}:
\begin{small}\begin{itemize}
\item Hazard analysis - 5.4.1(5), 5.4.2(8), 5.4.3(11), 5.4.4(13), 5.4.5(6), 5.4.6(3)
\item Security analysis - 5.4.1(6), 5.4.2(9), 5.4.3(12), 5.4.4(14), 5.4.5(7), 5.4.6(4)
\item Risk analysis - 5.4.1(7), 5.4.2(10), 5.4.3(13), 5.4.4(15), 5.4.5(8), 5.4.6(5)
\end{itemize}\end{small}\\
\hline
c. Generate structural and functional test data & Tests and inspections match
the severity of the product�s expected use & \underline{IEEE Std 1012-2004}:
\begin{small}\begin{itemize}
\item Component test case generation - 5.4.4(5)
\item Integration test case generation - 5.4.4(6)
\item System test case generation - 5.4.4(7)
\item Acceptance test case generation - 5.4.4(8)
\end{itemize}\end{small}\\
\hline \newpage
\multicolumn{3}{|c|}{\textit{\textbf{3. Construction}}}\\
\hline
a. Determine consistency with design & Insure that the product is reasonably
fit for its intended purposes & \underline{IEEE Std 1012-2004}:
\begin{small}\begin{itemize}
\item Traceability analysis - 5.4.4(1), 5.4.5(1)
\end{itemize}\end{small}\\
\hline
b. Determine adequacy of implementation & Secure the production of a safe
product & \underline{IEEE Std 1012-2004}:
\begin{small}\begin{itemize}
\item Component test execution = 5.4.4(12)
\end{itemize}\end{small}\\
\hline
c. Generate structural and functional test data & Tests and inspections match
the severity of the product�s expected use & 3.c.
\underline{IEEE Std 1012-2004}:
\begin{small}\begin{itemize}
\item Component test procedure generation - 5.4.4(9)
\item Integration test procedure generation - 5.4.4(10)
\item System test procedure generation - 5.4.4(11)
\item Acceptance test procedure generation - 5.4.5(2)
\end{itemize}\end{small}\\
\hline
d. Apply test data & Conclusive and effective to demonstrate the product�s
safety & \underline{IEEE Std 1012-2004}:
\begin{small}\begin{itemize}
\item Integration test execution - 5.4.5(3)
\item System test execution - 5.4.5(4)
\item Acceptance test execution - 5.4.5(5)
\end{itemize}\end{small}\\
\hline \newpage
\multicolumn{3}{|c|}{\textit{\textbf{4. Operation and Maintenance}}}\\
\hline
\multirow{4}{2in}{a. Reverify, commensurate with the level of redevelopment} & Make such tests and inspections during and after the process of manufacture &
\underline{IEEE Std 1012-2004}:
\begin{small}\begin{itemize}
\item Maintenance activities and tasks - 5.6
\end{itemize}\end{small}\\
\cline{2-3}
 & Test and inspect at the time(s) and place where the tests and inspections
will be effective & \underline{IEEE Std 1012-2004}:
\begin{small}\begin{itemize}
\item Installation and configuration audit - 5.4.6(1)
\item Installation checkout - 5.4.6(2)
\end{itemize}\end{small}\\
\cline{2-3}
 & Conduct tests and inspections following changes in the construction of the
product and/or field complaints of the defects & \underline{IEEE Std 1012-2004}:
\begin{small}\begin{itemize}
\item Evaluation of new constraints - 5.5.1(1)
\item Operating procedures evaluation - 5.5.1(2)
\end{itemize}\end{small}\\
\cline{2-3}
 & Discover latent hazards involved in the use of the product & \underline{IEEE Std 1012-2004}:
\begin{small}\begin{itemize}
\item Hazard analysis - 5.5.1(3)
\item Security analysis - 5.5.1(4)
\item Risk analysis - 5.5.1(5)
\end{itemize}\end{small}\\
\hline
\end{longtable}
\doublespace