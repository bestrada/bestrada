\chapter{Motivation}

Although software does offer the benefits of added flexibility, increased
functionality, and reduced costs, it provides unprecedented possibilities for
errors. Safety-critical systems have been on the bandwagon of using software in
their implementations for some time \cite{Graupe78,Hurtig94}, and disputes have
since then occurred regarding their stability \cite{Therac25,Maisel05}. The 
advent of such systems juxtaposed with the complexities of software breeds a new
set of concerns that do not easily map to traditional engineering standards.

Because of its unique nature, defects in software are inevitable and typically
more difficult to locate and handle than the physical flaws found in mechanical
components \cite{Parnas90}. Defects in software used in safety-critical
situations can be especially dangerous. The increasing use of software in
machines and the demand for more product functions adds complexity and more room
for error. Models to test, detect, and correct these errors exist and are
continually improving \cite{Parnas90}.

\section{Research Question}

Students of software engineering need to be aware of the societal effects that
the software they will write in industry is vulnerable to. How does our legal
system interact with the evolution and technical progress of software?
